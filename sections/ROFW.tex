\section{Read-Only Firmware}
% Because the RO Firmware is Read-Only, it is unable to updated during the life of the chromebook without dismantling the hardware.
% This is a double-edged sword for security purposes. 
% The RO FW is guaranteed to be the same code that shipped with the device, so one does not need to worry about it being compromised by attackers.
% However, any flaws in the shipped code are, by definition, unable to be patched and will remain a security vulnerability for the lifetime of the device.
% Because of these complications, the RO Firmware accomplishes the bare minimum of its responsibilities, leaving the more complicated tasks to the RW Firmware that is able to be patched ``In the Field''.

The Read-Only Firmware's purpose is to provide an initial configuration of the board's hardware and to verify the block of code that has been provided as the Read-Write Firmware.
The hardware required for this stage includes the TPM, SHA accelerator, Keyboard, and Display.

% THOUGHT: What if we could change the root key location *before* it got loaded in?

The first thing that RO Firmware is responsible for is loading in all of the Flash memory to RAM so that it is more easily accessible. 
After Flash memory is in RAM, data structures like the GBB and the Firmware Image are populated by reading the flashmap, which is always stored at a set location.
Once the data structures are populated and the hardware drivers are loaded, the Read-Only Firmware moves into the Vboot library to verify the next stage.


% RO FW destinations
If the RO Vboot fails, there are many options it can decide to take.
Almost all of the RO Vboot failures leads to a ``recovery reason'' being stored in Flash and then power cycling the system.
If the system is booted with a recovery reason, the RO Firmware will take the Recovery path and request to be restored from a Recovery USB. %TODO: see early stages and giant stage image

\subsubsection{TPM Properties Used}

The RO Firmware is responsible for controlling the TPM setup and ensuring that its properties hold. 
The TPM is responsible for attesting the boot flags and protecting against rollback attacks. 
Setting the TPM up requires that the TPM is lead through its self-check, and that the data zones required by Chrome exist within the TPM's state.

% Physical Presence
% http://www.trustedcomputinggroup.org/wp-content/uploads/Physical-Presence-Interface_1-30_0-52.pdf pg 19
Chrome disables the physical protection in the RO Firmware.
Physical presence is something that can be set high or low on a TPM either through software commands or a hardware wire. 
If PP is set high, then the ownership of the TPM can be changed and the re-provisioning operations become available.
For this reason, Chrome completely disables this option by permanently locking the TPM to a low PP\@.

% Rollback Attack
% TODO: Rollback attack visual
The other thing that the TPM protects from is Rollback Attacks.
In this situation a rollback attack is when older software with known vulnerabilities replaces newer, protected software in a malicious ``upgrade''.
This attack works against a naive implementation because the attacker is relying on an older version of ChromeOS that is available, and unmodified.
Because the attacker is using Google's software, the software is signed by Google's private key and will be accepted by the VBoot algorithm.
