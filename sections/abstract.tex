Formal Verification is an important tool for deeply analyzing security 
programs.
Formal Verification tools for software such as model checking are able to formally prove user specified assertions.
These proofs are created by exhaustively searching all corner cases, giving Formal Verification an edge over the traditional but less intensive Unit Testing framework. 
The benefits of Formal Verification are most seen in security programs where a bug or leak can have major real-world impacts.
Security programs are increasingly using hardware accelerators for faster computation, or hardware storage for tamper-proof guarantees.
The interactions between hardware and software are not captured in traditional model checkers, leading to possible errors in the verification. 
To showcase upcoming tools for verification across the hardware-software boundary, Formal Verification will be applied to the large, real world system of Google's Verified Boot. 
Verified Boot is an implementation of a secure bootloader with security guarantees across the hardware-software boundary. 
In order to use this system, it will need to be modified to run on Open-Source Hardware. 
In order to perform the verification, models for the hardware need to be created.
This hardware includes the Trusted Platform Module and a SHA accelerator.
These models will be created with the Instruction Level Abstraction toolchain. 
