% Title Page
\begin{center}
    \small{-- Independent Work Report Fall, 2016 --}
\end{center}

\vspace{1in}

\begin{center}
    \huge{Verified/Secure Boot: Formal Verification of Firmware \& Hardware in a large SoC} \\
%    \vspace{0.1in}
%    \large{A Management System for Undergraduate Resources}
\end{center}

\vspace{1in}

\begin{center}
    \Large{David Gilhooley} \\
    \vspace{0.1in}
    \large{Advisor: Sharad Malik} \\
\end{center}

\vspace{1in}

% Abstract
 \begin{center}
     \textbf{Abstract:}
 \end{center}

 \onehalfspacing
 Formal verification is an important tool for security verification because of its ability to do comprehensive corner case testing.
 It does this by providing either a guarantee that a specific property holds, or a counterexample that serves as a violation of the property.
 While formal verification has been increasingly popular in the checking of both software programs and hardware modules, checking both in a unified fashion is still under-developed.
 As computer architecture moves increasingly to specialized hardware for accelerators, security, and power saving purposes, this unified formal verification will become more more important.
 Real world systems like Chrome OS are making security claims about firmware that require verification across the hardware-firmware boundary. 
 This paper begins to investigate the tools available for this verification on a large scale platform that is widely used.



% Don't display header and start double spacing
\thispagestyle{empty}
