\sektion{2}{Approach}

Our app was designed with ease-of-use as the primary focus. In order to make the app as pleasant as possible for resources, administrators, and future developers, we defined the following to be guiding design philosophies:

%%% REGISTRATION %%%
\subsection{Intelligent registration and data collection}

Lengthy registration sequences are often notable deterrents for a product. The resource registration process should require as little information as possible:

Since we have access to data published by the University, the application should intelligently infer as much information as possible using those sources. In particular, objective information about a student, such as their name, class year, email address, and so on, can be collected from public websites such as Tigerbook \cite{tigerbook} given their netid. A student should only have to enter subjective information, such as how many hours they're willing to work each week, and their approximate schedule.

Ideally, we would be able to collect their grades automatically as well. However, we do not have the authority required to query grades. Getting proper permissions to do so would require lengthy negotiations with the registrar, or having course staff manually input grades. The former is undesirable, given the scope of this independent project, and the latter violates the premise of having ``intelligent'', automated data collection in the first place.

Instead, we concluded that having students upload their transcripts, and parsing grade information from those was an adequate solution. While parsing the transcript, we can ensure that we receive up-to-date and correct information, and it also removes the risk that we might not have the grades for a particular student. Therefore, a transcript should be the only other data required during registration.

%%% APPLICATION %%%
\subsection{Painless application and hiring process}

The job application and hiring processes should also be as simple as possible. Students should be able to choose which positions to be considered for at registration, and be automatically applied for those jobs. Hiring managers can then see who is interested in the position, and either reject them, extend an offer right away, or defer and request additional information from desirable candidates. In the last case, upon providing the additional information, be it in the form of an interview, an online form, or some other medium, the hiring manager should then have a second chance at either extending an offer, or rejecting the student. Upon receiving an offer, the resource can either accept or reject the invitation for employment.

It is critical to make the process as smooth as possible, as a student who feels burdened by the application process likely won't apply to many positions, or worse, decide not to use the app at all. Thus, special attention must be given to ensure that applying feels as natural and stressless as possible.

%%% MANAGEMENT TOOLS %%%
\subsection{Powerful and intuitive management tools}

Similarly, a hiring manager that feels encumbered by the management interface, or thinks the app lacks the necessary tools for the job, is also unlikely to use the site. Therefore, hiring managers should be able to view and sort candidates by attributes important to their specific position. Interviewers and student assistants should also be able to see necessary information about the candidates, but with sensitive details, such as grades, abstracted away.

There should also be a system for evaluating resources. When grading responses to an online form, or doing performance reviews at the end of a semester, the hiring managers will have to perform evaluations for many students at once. There should be a built-in system for administrators to this easily, efficiently, and collaboratively.

%%% API %%%
\subsection{RESTful API}

An intuitive API goes a long way when integrating with other apps or automating certain tasks. Having an accessible API means that other potential applications could use and benefit from our data, provided they have authentication to do so, of course. REST \cite{REST} is a popular design scheme for web APIs, and is both widely known, and very easy to interact with using \texttt{http}. We also felt it made the most sense given how easily the app lends itself to object-oriented construction.

%%% CODE DESIGN %%%
\subsection{Extensible Code Design}

Finally, we intend to extend this app to integrate with other applications, and anticipate needing to add new features quickly and often. Maintaining a code layout that is ``Open for extension, and closed for modification'', will help ensure the feature adding process is safe and painless for present and future developers.

