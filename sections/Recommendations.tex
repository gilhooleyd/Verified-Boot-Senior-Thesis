\chapter{Security Recommendations}

This section will combine the results in the last section to provide security
recommendations for Verified Boot.
While no security holes were found in Verified Boot, specific additions to the
Vboot process would reduce the surface of attack in the future.
Reducing the surface of attack would also allow other properties to be proven
which would be more desirable.

\section{Memory Locking}

The biggest improvement to Verified Boot would be the addition of Memory
Locking.
Other Secure Boot implementations add hardware which enables locking sections
of RAM\cite{elane}.
This feature would help prove the security of Verified Boot by isolating it from
any external repositories.

As mentioned in the Software chapter, Verified Boot makes function calls into
the Coreboot and Depthcharge repositories.
If the Firmware Image is not locked before these function calls are made, then
it is possible that the image is corrupted before the control returns back to
Verified Boot.
The Coreboot and Depthcharge repositories together contain 415,035 lines. 
Compare this to the 16,000 lines in the Vboot repository and it is unlikely that
Formal Verification could check that the other repositories do not corrupt the
Image.

%TODO: Add locking image

The flow in Image~\ref{} shows an example of how the hardware locking would be
implemented.
This would prevent any Time of Check Time of Use (TOCTOU) Vulnerabilities that
would exist outside of the Verified Boot library.
% TODO: Add another reference here
TOCTOU vulnerabilities are very common in secure programs\cite{tpm-toctou}.

This recommendation was brought to the attention of Google Engineer's through
the Chromium-OS email group.
Their response was that it is possible to implement under current hardware, and
it was recommended under NIST's ``Bios Protection Guidelines'', but it has not
been implemented yet.

Implementing this feature would allow a formal proof that the Image is not
altered after being verified but before being executed. 
Until this feature is implemented, there will always be the possibility of a
TOCTOU attack vector.

\section{TPM Key Storage}

% TODO: Do the hashing and encrypting on the TPM
% This would actually help because people could fake the hash
