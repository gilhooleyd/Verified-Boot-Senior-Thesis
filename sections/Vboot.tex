\documentclass[../report.tex]{subfiles}
 
\begin{document}
\onehalfspacing

\newpage
\section{Verified Boot}

Verified Boot (Vboot) is a cryptographic boot process that verifies the rest of the system code before running it, claiming to only run code that has been verified as signed by Google and untampered.
Vboot verifies both the system's firmware and operating system.
It is important for security reasons that Vboot itself is unmodified and cannot be removed by a hacker wishing to replace the system's firmware or OS undetected.
For this reason the Vboot code is stored and executed out of the Read Only Section of Flash which is designed to be unmodifiable without tampering with the system's hardware..
The Vboot process verifies a firmware image and kernel image in a two step process.
For Vboot to fulfill its security promises, it must only load and run images if the image was signed by Google and unmodified.
Images that have been calculated to be unsecure will not be loaded and the system will restart into a Recovery Mode.
Google has implemented different modes that Vboot can boot with and these modes alter the program flow and security guarantees.
In some modes, Verified Boot can even be removed from the system.
While seemingly counterintuitive to security, these modes have been outlined with their own security claims and properties.
I will be describing these modes in more detail in Section~\ref{sec:boot-modes}.


\subsection{Security Promises}

The main purpose of Chrome OS's verified boot is to provide relative security to the end user without sacrificing usability or functionality. 
In its mission statement, verified boot is designed against an ``opportunistic hacker''~\cite{vboot-design-doc}.
Vboot protects against vectors of attack that are relatively quick to exploit and it will realize the attack and nullify it on the next boot cycle.
An example of an attack that would be caught by Vboot would be the installation of a malicious kernel driver to act as a keylogger, or replacing the kernel with an older version with known vulnerabilities.

There are many attacks that Vboot is not able to recognize or protect against.
For example, Vboot does not make any promises to the safety of the system once the kernel is running and the user has control. 
Any attacks to the userspace of the Chromebook (for example the web-browser) would not affect the firmware or kernel image and would remain undetected.
However, userspace programs by definition have lower priorities and this would reduce the severity and influence of the attack.
Any attack to the kernel during runtime would remain until the system was powered down, which in certain situations could provide plenty of time for an attack to steal valuable data.



\subsection{Verified Boot Stages}

The boot process of a laptop is very complex and will include very different code across different laptops with different hardware.
In order to divide the code modularly and have Verified Boot be extensible for current and future platforms, Google has split the process into three main sections as seen in Figure~\ref{fig:vboot_stages_overview}.
These sections are referred to as Read-Only Firmware (RO FW), Read-Write Firmware (RW FW), and Kernel.
These sections are run in sequence with each section verifying the section that comes later.
In this way the root of trust starts from the Read-Only portion and builds until the system has booted completely.

\begin{figure}
\begin{subfigure}{.4\textwidth}
  \centering
  \includegraphics[width=1.0\linewidth]{vboot_stages_overview.png}
\end{subfigure}
\begin{subfigure}{.60\textwidth}
  \centering
  \includegraphics[width=1.0\linewidth]{vboot_stages_AB_recovery.png}
\end{subfigure}
\caption{Verified Boot is separated into hierarchical stages for clarity and security. This also shows the separate recovery path in case of a firmware issue.}
\label{fig:vboot_stages_overview}
\end{figure}

The RO Firmware runs first and as such is the root of trust for the entire platform.
It is responsible verifying the RW Firmware image and it contains all of the code and hardware drivers it needs to accomplish this task.
It verifies the RW Firmware Image signature using Google's main public key.
This public key is packaged with the RO Firmware and is also read only so it is guaranteed to be secure.
The RO Firmware is also responsible for handing the main control of Vboot, and this includes the transitions between Developer Mode, Safe Mode, and Recovery Mode.
These transitions will be described in more detail in Section~\ref{sec:boot-modes}.

% RW FW
The RO Firmware passes control to the RW Firmware once it can confirm that it has been signed by Google and is unmodified.
The RW Firmware contains the drivers for the rest of the hardware on the platform. 
It initializes the entire platform in preparation for booting and then it attempts to verify the kernel image.
The RW Firmware verifies the kernel using one of Google's public keys that is stored within the RW Firmware Image.
This public key had previously been verified as secure by the RO Firmware.
The RW Firmware Image is verifying the Kernel image off of its partition in the main hard drive.

\subsection{Boot Modes}\label{sec:boot-modes}

Vboot has three major boot states that influence the behavior of the boot. 
The first state is the Safe state which goes through the full process of Verified Boot normally.
The second state is the Recovery state which allows for a broken machine to format itself to its original state.
The third state is the Developer state and in this state the RSA signature is not verified.
The Developer state exists so that hobbyists can write their own firmware boot code and run operating systems other than Chrome OS\@.

Developer mode poses interesting security questions to Verified boot because it essentially disables the security guarantees and allows the system to move into an insecure state~\cite{developer-mode}. 
There are various security requirements around the Developer state transition.
First, a physical presence is required to fully complete the developer mode transition. 
This means that a person must sit at the computer once it has been rebooted and press a certain key combination (Control + D) when the developer mode screen appears.
The physical presence exists such that developer mode cannot be enabled through an off-site software attack without the user's knowledge.
The developer mode screen is referred to internally as the ``Scary Screen'', and its purpose is also to prevent users from being tricked into enabling developer mode by an external phishing party.
Once Developer Mode is enabled, the system can no longer claim guarantees about the boot process or its own secure storage.
In order to prevent an attacker from enabling Developer Mode so that they could read secure storage, Vboot wipes all secure storage on the transition into Developer Mode.
The secure storage that is wiped includes the RSA keys and various secrets stored in the TPM and the partion on disk where user data is stored.
Wiping this data on the transition is necessary to the confidentiality of the system and failure to do so is a security risk.

These precautions are also taken on the transition from Developer Mode to Safe Mode. 
If secure storage is left untouched moving from Developer Mode to Safe Mode, then an attacker would be able to place potentially malicious information in secure storage.
The assumptions that Google places on secure storage require that it can only be written to and read from by Google.
The platform will not be able to recognize if secure storage is in an insecure or malicious state and so wiping it is the easiest way to ensure that the platform has full control.
On the transition from Developer to Safe Mode, the system has to go through Recovery Mode.

Recovery Mode is responsible for getting the system back to a secure state.
Recovery Mode will be activated automatically when an error is recognized in Vboot.
This error could range from hardware failures to a corrupted image to a detected attack on the system.
When the system boots into Recovery it is for a Recovery Image stored on external memory, either on an SD card or a USB drive.
The path for recovery uses a different RSA key for modularity and this recovery key is also stored in Read Only Flash in order to be unmodifiable.
Once the recovery image is verified to be secure and unmodified, the image is loaded off of the external storage and it is used to replace the images currently stored on the device.
Recovery mode is also responsible for wiping secure memory, including the user's disk partition and TPM's secure storage.
If Recovery completes successfully, the system is in a secure state and Vboot can then continue safely.
From a user perspective, Recovery is an easy way to wipe the device and restore it to Factory Settings.
Recovery is also the only way to roll the device back to an early image, and this rollback is sometimes necessary if there are known problems with a newly uploaded image.

\subsection{Data Structures}\label{sec:data-structures}

To start understanding the Vboot verification process, it is necessary to talk about the data structures that are used throughout. 
These data structures are populated by the Firmware or Kernel Image that is attempting to be verified.
Through this section I will start at the highest hierarchical level of data structure then explain the structures that are contained within.

\subsubsection{Firmware/Kernel Image}

\begin{figure}
\begin{subfigure}{.5\textwidth}
  \centering
  \includegraphics[width=1.0\linewidth]{fw_image.png}
\end{subfigure}%
\begin{subfigure}{.5\textwidth}
  \centering
  \includegraphics[width=1.0\linewidth]{kernel_image.png}
\end{subfigure}
\caption{Layout of the Firmware and Kernel Images~\cite{vboot-data-structures}}
\label{fig:vboot_images}
\end{figure}

The actual image is not a data structure but a chunk of data that is stored contiguously in non-volatile memory.
The image structure, as seen in Figure~\ref{fig:vboot_images}, consists of three parts: a key block, a preamble, and the main body of the image.
The key block is verified first, and it is verified by the main RSA key stored in Read Only memory.
Once the key block has been shown to be safe, the RSA key located within will be used to verify the preamble.
The preamble contains the hash of the image's body.
The image's body contains the code that is going to be run next and it is checked against the hash stored in the preamble.
If the preamble is for the Firmware image then it will contain the RSA key used to verify the key block of the Kernel image.
If the preamble is for the Kernel image then it will contain information about the location of the Kernel's bootloader, like where it exists in the Kernel image and how it should be loaded into RAM\@.

\subsubsection{Key Blocks}

\begin{figure}
\begin{subfigure}{.5\textwidth}
  \centering
  \includegraphics[width=1.0\linewidth]{vboot_keyblock.png}
\end{subfigure}
\begin{subfigure}{.20\textwidth}
  \centering
  \includegraphics[width=1.0\linewidth]{vbpublickey.png}
\end{subfigure}
\begin{subfigure}{.20\textwidth}
  \centering
  \includegraphics[width=1.0\linewidth]{vbsignature.png}
\end{subfigure}
\caption{The Keyblock data structure and Metadata for Keys and Signatures}
\label{fig:vboot_keyblock}
\end{figure}

The Key block is the first part of the image that is validated and it is used to validate the rest of the image.
The Key block is the data structure that allows a hierarchy of RSA keys to be used during Vboot.
Figure~\ref{fig:vboot_keyblock} shows the structure of the keyblock. 
The key block flags that are mentioned are used to determine which mode of Vboot the Keyblock is valid in. 
There are 4 possible boot modes corresponding to the combination of the two binary options, Developer and Recovery.
% info located in vboot_struct.h

Within the keyblock there exists data structures for a public key and a signature.
Google has added the ability to change their encryption strength.
They have added support for RSA 1024, 2048, 4096, 8192 and for SHA 1, 256, 512, for a total of 12 different possible algorithm combinations.

\subsubsection{Google Binary Block}

% info found in gbb_header.h
The Google Binary Block (GBB) is a part of Vboot's Root of Trust.
It is a data structure stored in Read-Only memory that is initialized and configured in the factory at the laptop's creation.
It contains the Root and Recovery RSA keys, the Hardware ID (HWID) used to identify the specific laptop make and model, a host of flags that affect boot operation, and the bitmaps used for various boot screens.


\subsection{Code Organization}

Like almost all modern operating systems, Chrome OS is written in C.
Like Linux, Chrome OS is maintained using Git for version control. 
Git is a diff-based, non-centralized version control system that makes it easy for programmers to share code, rollback changes, and maintain separate branches of the same codebase~\cite{git}.
Google has built a tool called ``repo'' that is used on top of git~\cite{repo}. 
Repo is a tool to manipulate multiple code repositories. 
It's primary benefit is that it allows a company to specify how multiple git repositories should be installed and placed within a given file-system.
This is both helpful and necessary as Chrome OS consists of over one thousand different external and internal repositories. 

Coreboot, vboot\_reference, depthcharge, are the repositories used for the firmware boot process.
The flow of Verfied Boot through the repos can be seen in Figure~\ref{fig:code_repos} and the purpose of each repo is described below.

\begin{figure}
  \centering
  \includegraphics[width=0.8\linewidth]{code_repos}
  \caption{ChromeOS's boot flow goes through Coreboot, Depthcharge, and the Vboot Library twice for Firmware and Kernel verification}
  \label{fig:code_repos}
\end{figure}

\subsubsection{Coreboot}

Coreboot is a fully Open-Sourced alternative to traditional BIOS implementations.~\cite{coreboot}
It is lightweight and is configured to implement the full standard of the Unified Extensible Firmware Interface (UEFI).
Google has chosen Coreboot because of its small code footprint, full extensibility, and the fact that it is available freely as an Open Source project.

The Coreboot code is responsible for doing very early initialization code on the main CPU\@. 
% This includes things like setting up the GPIO pins, enabling hardware interrupts, setting up a large stack in RAM, and providing driver callbacks to the payload that it will eventually call.
Coreboot is setup so that once a baseline level of initialization is complete, it passes control to another section of code called a ``payload''~\cite{coreboot-payload}.
This payload is responsible for initializing the more specialized drivers, and the concept of a payload means that Google can keep support more hardware without altering the Coreboot source code.
The payload that Coreboot calls to further initialize the Chromebook is Depthcharge.
% Init done in src/mainboard/google/{link}/romstage.c

% TODO: Who is responsible for loading the Flashmap into MMIO? 

\subsubsection{DepthCharge}

Depthcharge contains the minimum number of drivers necessary for the virtual boot to work successfully~\cite{depthcharge-codebase}. 
The important drivers include the TPM, I2C, the EC, SPI and SPI Flash, and the display~\cite{depthcharge-slides}.
Once the drivers are initialized successfully, Depthcharge creates the necessary structures for Vboot and then passes control into the Vboot library.

Depthcharge is Google's repo that holds platform dependent code.
The different branches in the repository hold different drivers as they are needed for each of the Chromebook's hardware.

\subsubsection{Vboot\_reference}

The vboot\_reference repo contains all of the control and algorithms for the vboot process~\cite{vboot-codebase}.
The repo is designed such that it does not rely on any knowledge about the platform.
If a function requires usage of a driver or something that is board specific, it will make a callback into Depthcharge which will provide the relevant information.


\end{document}
